\documentclass[11pt]{article}

% load some asm stuff -
\usepackage{amssymb}
\usepackage{amsmath}
\usepackage{amsthm}
%\usepackage{palatino,lettrine}
\usepackage{fancyhdr}
\usepackage{epsfig}
\usepackage[round,comma,sort]{natbib}
\usepackage{simplemargins}
\usepackage{setspace}
\usepackage[margin=0pt,font=small,labelfont=bf]{caption}

\bibliographystyle{plos2009}

% Set the size
%\textwidth = 6.75 in
%\textheight = 9.75 in
%\oddsidemargin = 0.0 in
%\evensidemargin = 0.0 in
%\topmargin = 0.01 in
%\headheight = 0.0 in
%\headsep = 0.25 in
%\parskip = 0.15in
\doublespace

\setallmargins{1in}

\newtheorem{example}{Example}[section]
\newtheorem{thm}{Theorem}[section]
\newtheorem{property}{Property}[section]

\theoremstyle{definition}
\newtheorem{defn}[thm]{Definition}

\makeatletter
\renewcommand\subsection{\@startsection
	{subsection}{2}{0mm}
	{-0.05in}
	{-0.5\baselineskip}
	{\normalfont\normalsize\bfseries}}
\renewcommand\subsubsection{\@startsection
	{subsubsection}{2}{0mm}
	{-0.05in}
	{-0.5\baselineskip}
	{\normalfont\normalsize\itshape}}
\renewcommand\paragraph{\@startsection
	{paragraph}{2}{0mm}
	{-0.05in}
	{-0.5\baselineskip}
	{\normalfont\normalsize\itshape}}
\makeatother
\linespread{1.2}

\fancypagestyle{proposal}{\fancyhf{}%
	\fancyhead[RO,LE]{\thepage}%
	\fancyhead[LO,RE]{ChE 525 Analysis of Well-Mixed Fed-batch Cultures}%
	\renewcommand\headrulewidth{1pt}}
\pagestyle{proposal}

% Single space'd bib -
\setlength\bibsep{0pt}

\renewcommand{\rmdefault}{phv}\renewcommand{\sfdefault}{phv}

%\newboxedtheorem[boxcolor=black, background=gray!5,titlebackground=orange!20,titleboxcolor = black]{color_box_example}{Example}{test}

% Change the number format in the ref list -
\renewcommand{\bibnumfmt}[1]{#1.}

% Change Figure to Fig.
\renewcommand{\figurename}{Fig.}

%Joycelyn Chan, Joshua Lequieu, Michael Paull, Chidanand Balaji, Ryan Tasseff
%Our derivation follows closely the earlier development of Fredrickson \citep{Fredrickson:1976fk}.

% Begin ...
\begin{document}

%\begin{titlepage}
{\par\centering\textbf{\Large Analysis of Well-Mixed Fed-Batch Cultures using Unstructured Models}}
\vspace{0.2in}
{\par \centering \large{Jeffrey D. Varner$^{*}$}}
\vspace{0.05in}
{\par \centering \large{School of Chemical Engineering$^{*}$}}
{\par \centering \large{Purdue University, West Lafayette IN 47907}}
\vspace{0.1in}
{\par \centering \small{Copyright \copyright\ Jeffrey Varner 2016. All Rights Reserved.}}\\

%\end{titlepage}
\date{}
\thispagestyle{empty}

\setcounter{page}{1}

%material and energy balances around the different processes cells do. For example, understanding how the abundance of raw materials in a bioreactor influences
%cell growth, or the production of valuable protein or small molecule products requires a materials balances around the major components of the system.
%The production of valuable small molecule or protein products requires large connected intracellular reaction networks that produce or consume energy.
%Thus, to understand the operation of biochemical systems and ultimately to manipulate them for societal gain,

\section*{Introduction}
Fed-batch cultures are the most complex of the three modes of operating a bioreactor. Fed-batch cultures are dynamic and have volume change.
Like a continuous culture, feb-batch cultures have an input feed stream into the reactor.
However, unlike chemostats, there is no outflow from the reaction vessel in a fed-batch reactor.
Thus, the working volume in the reactor increases over time diluting the contents of the vessel.
In this lecture, we'll develop a mathematical description of
fed-batch cultures using the general material balances, and then explore how the performance of the fed-batch culture can be optimized.

\subsection*{General model equations for fed-batch cultures.}
Let's start with the general material balances we derived previously:
\begin{eqnarray}\label{eqn-metabolite-dilution-dynamic}
	\frac{dC_{j}}{dt} &=& \sum_{s~=~1}^{\mathcal{S}}v_{s}D_{s}C_{j,s} + \left(\sum_{r~=~1}^{\mathcal{R}}\sigma_{jr}\hat{r}_{r}\right) + \left(\sum_{k~=~1}^{\mathcal{T}}\tau_{j,k}q_{k}\right)X  - \frac{C_{j}}{V}\frac{dV}{dt}\qquad j=1,2,\dots,\mathcal{M}\\
	\frac{dX}{dt} &=& \sum_{s~=~1}^{\mathcal{S}}v_{s}D_{s}X_{s}+\left(\mu - k_{d}\right)X - \frac{X}{V}\frac{dV}{dt}\\
	\frac{dV}{dt} &=& \sum_{s~=~1}^{\mathcal{S}}v_{s}\frac{\rho_{s}}{\rho}F_{s} - \frac{V}{\rho}\frac{d\rho}{dt}
\end{eqnarray}where the quantity $D_{s}$,  called a \textit{dilution~rate} (hr$^{-1}$), is given as:
\begin{equation}
	D_{s} \equiv \frac{F_{s}}{V}\qquad s=1,2,\dots,\mathcal{S}
\end{equation}The quantity $C_{j}$ denotes the concentration of the jth extracellular metabolite, $V$ denotes the working volume of the culture and $X$ denotes the cellmass.
In a fed-batch culture, there are in-flows, but no out-flow from the culture vessel, thus $D_{s^{-}} = 0$ where $s^{-}$ denote the set of outflow streams.
Because there is no out-flow, there is a volume change ($dV/dt > 0$), and all the dilution terms in the material balances are non-negative:
\begin{eqnarray}\label{eqn-metabolite-batch}
	\frac{dC_{j}}{dt} &=& \sum_{s~=~1}^{\mathcal{S^{+}}}v_{s}D_{s}C_{j,s} + \left(\sum_{r~=~1}^{\mathcal{R}}\sigma_{jr}\hat{r}_{r}\right) + \left(\sum_{k~=~1}^{\mathcal{T}}\tau_{j,k}q_{k}\right)X - \frac{C_{j}}{V}\frac{dV}{dt} \qquad j=1,2,\dots,\mathcal{M}\\
	\frac{dX}{dt} &=& \sum_{s~=~1}^{\mathcal{S^{+}}}v_{s}D_{s}X_{s} + \left(\mu - k_{d}\right)X - - \frac{X}{V}\frac{dV}{dt}\\
	\frac{dV}{dt} &=& \sum_{s~=~1}^{\mathcal{S^{+}}}v_{s}\frac{\rho_{s}}{\rho}F_{s} - \frac{V}{\rho}\frac{d\rho}{dt}
\end{eqnarray}

\subsection*{Analysis of a simple fed-batch culture.}
To better understand the dynamics of a fed-batch culture, let's simplify the general equations by assuming a Monod growth model \citep{Legout:2010aa}, a single limiting nutrient $S$, a single sterile input feed stream (s=1)
with similar density to the working volume, no density change as a function of time, stable substrate and product, and no maintenance utilization of substrate.
With these assumptions the general fed-batch balances reduce to:
\begin{eqnarray}\label{eqn-metabolite-batch-simple}
	\frac{dS}{dt} &=& D_{1}S_{1} - \frac{1}{Y_{X/S}^{*}}\mu X - \frac{1}{Y_{P/S}}q_{p} X - \frac{S}{V}\frac{dV}{dt}\\
	\frac{dP}{dt} &=& D_{1}P_{1} + q_{p}X - \frac{P}{V}\frac{dV}{dt}\\
	\frac{dX}{dt} &=& \left(\mu - k_{d}\right)X - \frac{X}{V}\frac{dV}{dt}\\
	\frac{dV}{dt} &=& F_{1}\left(t\right)
\end{eqnarray}
If we substitute the volume balance into the substrate, product and cellmass balances (and drop the stream index on the dilution rate), we arrive at the simplified system of equations:
\begin{eqnarray}\label{eqn-metabolite-fedbatch-simple}
	\frac{dS}{dt} &=& D\left(S_{1} - S\right) - \frac{1}{Y_{X/S}^{*}}\mu X - \frac{1}{Y_{P/S}}q_{p} X\\
	\frac{dP}{dt} &=& D\left(P_{1}- P\right) + q_{p}X\\
	\frac{dX}{dt} &=& \left(\mu - k_{d}\right)X - DX \\
	\frac{dV}{dt} &=& F\left(t\right)
\end{eqnarray}
where $\mu$ is given by:
\begin{equation}\label{eqn-monod-growth-model}
	\mu = \mu_{g}^{max}\left(\frac{S}{K_{g} + S}\right)
\end{equation} and we assume the Luedeking and Piret model for product formation \citep{Luedeking:2000aa}:
\begin{eqnarray}
	q_{p} = \alpha~\mu+\beta
\end{eqnarray}
The cellmass, substrate, product and volume balances are coupled nonlinear differential equations which can be solved numerically using common packages such as MATLAB or JULIA \citep{BEKS14}.

\subsubsection*{Growth associated production formation.}
\subsubsection*{Non-growth associated production formation.}
\subsubsection*{Mixed product formation.}

\subsection*{How do we measure the performance of a bacterial culure?}

\bibliography{Notes}
\end{document}
